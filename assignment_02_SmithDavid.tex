\documentclass{article}\usepackage[]{graphicx}\usepackage[]{xcolor}
% maxwidth is the original width if it is less than linewidth
% otherwise use linewidth (to make sure the graphics do not exceed the margin)
\makeatletter
\def\maxwidth{ %
  \ifdim\Gin@nat@width>\linewidth
    \linewidth
  \else
    \Gin@nat@width
  \fi
}
\makeatother

\definecolor{fgcolor}{rgb}{0.345, 0.345, 0.345}
\newcommand{\hlnum}[1]{\textcolor[rgb]{0.686,0.059,0.569}{#1}}%
\newcommand{\hlstr}[1]{\textcolor[rgb]{0.192,0.494,0.8}{#1}}%
\newcommand{\hlcom}[1]{\textcolor[rgb]{0.678,0.584,0.686}{\textit{#1}}}%
\newcommand{\hlopt}[1]{\textcolor[rgb]{0,0,0}{#1}}%
\newcommand{\hlstd}[1]{\textcolor[rgb]{0.345,0.345,0.345}{#1}}%
\newcommand{\hlkwa}[1]{\textcolor[rgb]{0.161,0.373,0.58}{\textbf{#1}}}%
\newcommand{\hlkwb}[1]{\textcolor[rgb]{0.69,0.353,0.396}{#1}}%
\newcommand{\hlkwc}[1]{\textcolor[rgb]{0.333,0.667,0.333}{#1}}%
\newcommand{\hlkwd}[1]{\textcolor[rgb]{0.737,0.353,0.396}{\textbf{#1}}}%
\let\hlipl\hlkwb

\usepackage{framed}
\makeatletter
\newenvironment{kframe}{%
 \def\at@end@of@kframe{}%
 \ifinner\ifhmode%
  \def\at@end@of@kframe{\end{minipage}}%
  \begin{minipage}{\columnwidth}%
 \fi\fi%
 \def\FrameCommand##1{\hskip\@totalleftmargin \hskip-\fboxsep
 \colorbox{shadecolor}{##1}\hskip-\fboxsep
     % There is no \\@totalrightmargin, so:
     \hskip-\linewidth \hskip-\@totalleftmargin \hskip\columnwidth}%
 \MakeFramed {\advance\hsize-\width
   \@totalleftmargin\z@ \linewidth\hsize
   \@setminipage}}%
 {\par\unskip\endMakeFramed%
 \at@end@of@kframe}
\makeatother

\definecolor{shadecolor}{rgb}{.97, .97, .97}
\definecolor{messagecolor}{rgb}{0, 0, 0}
\definecolor{warningcolor}{rgb}{1, 0, 1}
\definecolor{errorcolor}{rgb}{1, 0, 0}
\newenvironment{knitrout}{}{} % an empty environment to be redefined in TeX

\usepackage{alltt}
\usepackage[sc]{mathpazo}
\renewcommand{\sfdefault}{lmss}
\renewcommand{\ttdefault}{lmtt}
\usepackage[T1]{fontenc}
\usepackage{geometry}
\geometry{verbose,tmargin=2.5cm,bmargin=2.5cm,lmargin=2.5cm,rmargin=2.5cm}
\setcounter{secnumdepth}{2}
\setcounter{tocdepth}{2}
\usepackage[unicode=true,pdfusetitle,
 bookmarks=true,bookmarksnumbered=true,bookmarksopen=true,bookmarksopenlevel=2,
 breaklinks=false,pdfborder={0 0 1},backref=false,colorlinks=false]
 {hyperref}
\hypersetup{
 pdfstartview={XYZ null null 1}}

\makeatletter
%%%%%%%%%%%%%%%%%%%%%%%%%%%%%% User specified LaTeX commands.
\renewcommand{\textfraction}{0.05}
\renewcommand{\topfraction}{0.8}
\renewcommand{\bottomfraction}{0.8}
\renewcommand{\floatpagefraction}{0.75}

\makeatother
\IfFileExists{upquote.sty}{\usepackage{upquote}}{}
\begin{document}



\title{\title{\title{\title{\title{\title{}}}}}}



\maketitle
The results below are generated from an R script.

\begin{knitrout}
\definecolor{shadecolor}{rgb}{0.969, 0.969, 0.969}\color{fgcolor}\begin{kframe}
\begin{alltt}
\hlcom{# Assignment: ASSIGNMENT 2}
\hlcom{# Name: Smith, David}
\hlcom{# Date: 20d23-06-18}

\hlcom{## Check your current working directory using `getwd()`}
\hlkwd{getwd}\hlstd{()}
\end{alltt}
\begin{verbatim}
## [1] "F:/GitLab-Projects/Bellevue/local-smith_dsc520"
\end{verbatim}
\begin{alltt}
\hlcom{## List the contents of the working directory with the `dir()` function}
\hlkwd{dir}\hlstd{()}
\end{alltt}
\begin{verbatim}
##  [1] "assignment_02_SmithDavid.log" "assignment_02_SmithDavid.pdf"
##  [3] "assignment_02_SmithDavid.tex" "assignments"                 
##  [5] "completed"                    "data"                        
##  [7] "LICENSE"                      "local-smith_dsc520.Rproj"    
##  [9] "README.md"                    "RMarkdown.md"
\end{verbatim}
\begin{alltt}
\hlcom{## If the current directory does not contain the `data` directory, set the}
\hlcom{## working directory to project root folder (the folder should contain the `data` directory}
\hlcom{## Use `setwd()` if needed}
\hlkwd{setwd}\hlstd{(}\hlstr{"F:/GitLab-Projects/Bellevue/smith-dsc520"}\hlstd{)}

\hlcom{## Load the file `data/tidynomicon/person.csv` to `person_df1` using `read.csv`}
\hlcom{## Examine the structure of `person_df1` using `str()`}
\hlstd{person_df1} \hlkwb{<-} \hlkwd{read.csv}\hlstd{(}\hlstr{"F:/GitLab-Projects/Bellevue/smith-dsc520/data/tidynomicon/person.csv"}\hlstd{)}
\hlkwd{str}\hlstd{(person_df1)}
\end{alltt}
\begin{verbatim}
## 'data.frame':	5 obs. of  3 variables:
##  $ person_id    : chr  "dyer" "pb" "lake" "roe" ...
##  $ personal_name: chr  "William" "Frank" "Anderson" "Valentina" ...
##  $ family_name  : chr  "Dyer" "Pabodie" "Lake" "Roerich" ...
\end{verbatim}
\begin{alltt}
\hlcom{## R interpreted names as factors, which is not the behavior we want}
\hlcom{## Load the same file to person_df2 using `read.csv` and setting `stringsAsFactors` to `FALSE`}
\hlcom{## Examine the structure of `person_df2` using `str()`}
\hlstd{person_df2} \hlkwb{<-} \hlkwd{read.csv}\hlstd{(}\hlstr{"F:/GitLab-Projects/Bellevue/smith-dsc520/data/tidynomicon/person.csv"}\hlstd{,} \hlkwc{stringsAsFactors}\hlstd{=}\hlnum{FALSE}\hlstd{)}
\hlkwd{str}\hlstd{(person_df2)}
\end{alltt}
\begin{verbatim}
## 'data.frame':	5 obs. of  3 variables:
##  $ person_id    : chr  "dyer" "pb" "lake" "roe" ...
##  $ personal_name: chr  "William" "Frank" "Anderson" "Valentina" ...
##  $ family_name  : chr  "Dyer" "Pabodie" "Lake" "Roerich" ...
\end{verbatim}
\begin{alltt}
\hlcom{## Read the file `data/scores.csv` to `scores_df`}
\hlcom{## Display summary statistics using the `summary()` function}
\hlstd{scores_df} \hlkwb{<-} \hlkwd{read.csv}\hlstd{(}\hlstr{"F:/GitLab-Projects/Bellevue/smith-dsc520/data/scores.csv"}\hlstd{)}
\hlkwd{summary}\hlstd{(scores_df)}
\end{alltt}
\begin{verbatim}
##      Count           Score         Section         
##  Min.   :10.00   Min.   :200.0   Length:38         
##  1st Qu.:10.00   1st Qu.:300.0   Class :character  
##  Median :10.00   Median :322.5   Mode  :character  
##  Mean   :14.47   Mean   :317.5                     
##  3rd Qu.:20.00   3rd Qu.:357.5                     
##  Max.   :30.00   Max.   :395.0
\end{verbatim}
\begin{alltt}
\hlcom{## Load the `readxl` library}
\hlkwd{library}\hlstd{(readxl)}

\hlcom{## Using the excel_sheets() function from the `readxl` package,}
\hlcom{## list the worksheets from the file `data/G04ResultsDetail2004-11-02.xls`}
\hlstd{voter_turnout_df} \hlkwb{<-} \hlkwd{excel_sheets}\hlstd{(}\hlstr{"F:/GitLab-Projects/Bellevue/smith-dsc520/data/G04ResultsDetail2004-11-02.xls"}\hlstd{)}
\hlstd{voter_turnout_df}
\end{alltt}
\begin{verbatim}
##  [1] "Instructions"          "Voter Turnout"         "President"            
##  [4] "House of Rep"          "Co Clerk"              "Co Reg Deeds"         
##  [7] "Co Public Defender"    "Co Comm 1"             "Co Comm 3"            
## [10] "Co Comm 5"             "Co Comm 7"             "St Bd of Ed 2"        
## [13] "St Bd of Ed 4"         "Legislature 5"         "Legislature 7"        
## [16] "Legislature 9"         "Legislature 11"        "Legislature 13"       
## [19] "Legislature 23"        "Legislature 31"        "Legislature 39"       
## [22] "MCC 1"                 "MCC 2"                 "MCC 3"                
## [25] "MCC 4"                 "OPPD"                  "MUD"                  
## [28] "NRD 3"                 "NRD 5"                 "NRD 7"                
## [31] "NRD 9"                 "OPS 2"                 "OPS 4"                
## [34] "OPS 6"                 "OPS 8"                 "OPS 10"               
## [37] "OPS 11"                "OPS 12"                "ESU 2"                
## [40] "ESU 3"                 "Arlington Sch 24"      "Bennington Sch 59"    
## [43] "Elkhorn Sch 10"        "Fremont Sch 1"         "Ft Calhoun Sch 3"     
## [46] "Gretna Sch 37"         "Millard Sch 17"        "Ralston Sch 54"       
## [49] "Valley Sch 33"         "Waterloo Sch 11"       "Bennington Mayor"     
## [52] "Elkhorn Mayor"         "Valley Mayor"          "Ralston Mayor"        
## [55] "Ralston Library Bd"    "Bennington City Cnc 1" "Bennington City Cnc 2"
## [58] "Elkhorn City Cnc A"    "Elkhorn City Cnc B"    "Elkhorn City Cnc C"   
## [61] "Ralston City Cnc 1"    "Ralston City Cnc 2"    "Ralston City Cnc 6"   
## [64] "Waterloo Bd Trustees"  "Valley City Cnc"       "Amendment 1"          
## [67] "Amendment 2"           "Amendment 3"           "Amendment 4"          
## [70] "Initiative 417"        "Initiative 418"        "Initiative 419"       
## [73] "Initiative 420"
\end{verbatim}
\begin{alltt}
\hlcom{## Using the `read_excel` function, read the Voter Turnout sheet}
\hlcom{## from the `data/G04ResultsDetail2004-11-02.xls`}
\hlcom{## Assign the data to the `voter_turnout_df1`}
\hlcom{## The header is in the second row, so make sure to skip the first row}
\hlcom{## Examine the structure of `voter_turnout_df1` using `str()`}

\hlstd{voter_turnout_df1} \hlkwb{<-} \hlkwd{read_excel}\hlstd{(}\hlstr{"F:/GitLab-Projects/Bellevue/smith-dsc520/data/G04ResultsDetail2004-11-02.xls"}\hlstd{)}
\hlkwd{str}\hlstd{(voter_turnout_df1)}
\end{alltt}
\begin{verbatim}
## tibble [19 x 1] (S3: tbl_df/tbl/data.frame)
##  $ Douglas County Election Commission: chr [1:19] "November 2, 2004 Presidential General Election" "Official Results" NA NA ...
\end{verbatim}
\begin{alltt}
\hlcom{## Using the `read_excel()` function, read the Voter Turnout sheet}
\hlcom{## from `data/G04ResultsDetail2004-11-02.xls`}
\hlcom{## Skip the first two rows and manually assign the columns using `col_names`}
\hlcom{## Use the names "ward_precint", "ballots_cast", "registered_voters", "voter_turnout"}
\hlcom{## Assign the data to the `voter_turnout_df2`}
\hlcom{## Examine the structure of `voter_turnout_df2` using `str()`}
\hlstd{voter_turnout_df2} \hlkwb{<-} \hlkwd{read_excel}\hlstd{(}\hlstr{"F:/GitLab-Projects/Bellevue/smith-dsc520/data/G04ResultsDetail2004-11-02.xls"}\hlstd{,}\hlkwc{skip}\hlstd{=}\hlnum{2}\hlstd{)}
\hlkwd{str}\hlstd{(voter_turnout_df2)}
\end{alltt}
\begin{verbatim}
## tibble [17 x 1] (S3: tbl_df/tbl/data.frame)
##  $ Official Results: chr [1:17] NA NA NA "To view the results of a particular race, click on the tab for the sheet you wish to view." ...
\end{verbatim}
\begin{alltt}
\hlcom{## Load the `DBI` library}
\hlkwd{library}\hlstd{(DBI)}

\hlcom{## Create a database connection to `data/tidynomicon/example.db` using the dbConnect() function}
\hlcom{## The first argument is the database driver which in this case is `RSQLite::SQLite()`}
\hlcom{## The second argument is the path to the database file}
\hlcom{## Assign the connection to `db` variable}
\hlstd{db} \hlkwb{<-} \hlkwd{dbConnect}\hlstd{(RSQLite}\hlopt{::}\hlkwd{SQLite}\hlstd{(),}\hlstr{"data/tidynomicon/example.db"}\hlstd{)}

\hlcom{## Query the Person table using the `dbGetQuery` function and the}
\hlcom{## `SELECT * FROM PERSON;` SQL statement}
\hlcom{## Assign the result to the `person_df` variable}
\hlcom{## Use `head()` to look at the first few rows of the `person_df` dataframe}
\hlstd{person_df} \hlkwb{<-} \hlkwd{dbGetQuery}\hlstd{(db,}\hlstr{"SELECT * FROM PERSON;"}\hlstd{)}
\hlkwd{head}\hlstd{(person_df)}
\end{alltt}
\begin{verbatim}
##   person_id personal_name family_name
## 1      dyer       William        Dyer
## 2        pb         Frank     Pabodie
## 3      lake      Anderson        Lake
## 4       roe     Valentina     Roerich
## 5  danforth         Frank    Danforth
\end{verbatim}
\begin{alltt}
\hlcom{## List the tables using the `dbListTables()` function}
\hlcom{## Assign the result to the `table_names` variable}
\hlstd{table_names} \hlkwb{<-} \hlkwd{dbListTables}\hlstd{(db)}
\hlstd{table_names}
\end{alltt}
\begin{verbatim}
## [1] "Measurements" "Person"       "Site"         "Visited"
\end{verbatim}
\begin{alltt}
\hlcom{## Read all of the tables at once using the `lapply` function and assign the result to the `tables` variable}
\hlcom{## Use `table_names`, `dbReadTable`, and `conn = db` as arguments}
\hlcom{## Print out the tables}
\hlstd{tables} \hlkwb{<-} \hlkwd{lapply}\hlstd{(table_names,dbReadTable,}\hlkwc{conn} \hlstd{= db)}
\end{alltt}


{\ttfamily\noindent\color{warningcolor}{\#\# Warning: Column `reading`: mixed type, first seen values of type real, coercing other values of type string}}\begin{alltt}
\hlstd{tables}
\end{alltt}
\begin{verbatim}
## [[1]]
##    visit_id person_id quantity reading
## 1       619      dyer      rad    9.82
## 2       619      dyer      sal    0.13
## 3       622      dyer      rad    7.80
## 4       622      dyer      sal    0.09
## 5       734        pb      rad    8.41
## 6       734      lake      sal    0.05
## 7       734        pb     temp  -21.50
## 8       735        pb      rad    7.22
## 9       735      <NA>      sal    0.06
## 10      735      <NA>     temp  -26.00
## 11      751        pb      rad    4.35
## 12      751        pb     temp  -18.50
## 13      751      lake      sal    0.00
## 14      752      lake      rad    2.19
## 15      752      lake      sal    0.09
## 16      752      lake     temp  -16.00
## 17      752       roe      sal   41.60
## 18      837      lake      rad    1.46
## 19      837      lake      sal    0.21
## 20      837       roe      sal   22.50
## 21      844       roe      rad   11.25
## 
## [[2]]
##   person_id personal_name family_name
## 1      dyer       William        Dyer
## 2        pb         Frank     Pabodie
## 3      lake      Anderson        Lake
## 4       roe     Valentina     Roerich
## 5  danforth         Frank    Danforth
## 
## [[3]]
##   site_id latitude longitude
## 1    DR-1   -49.85   -128.57
## 2    DR-3   -47.15   -126.72
## 3   MSK-4   -48.87   -123.40
## 
## [[4]]
##   visit_id site_id visit_date
## 1      619    DR-1 1927-02-08
## 2      622    DR-1 1927-02-10
## 3      734    DR-3 1930-01-07
## 4      735    DR-3 1930-01-12
## 5      751    DR-3 1930-02-26
## 6      752    DR-3       <NA>
## 7      837   MSK-4 1932-01-14
## 8      844    DR-1 1932-03-22
\end{verbatim}
\begin{alltt}
\hlcom{## Use the `dbDisconnect` function to disconnect from the database}
\hlkwd{dbDisconnect}\hlstd{(db)}

\hlcom{## Import the `jsonlite` library}
\hlkwd{library}\hlstd{(jsonlite)}

\hlcom{## Convert the scores_df dataframe to JSON using the `toJSON()` function}
\hlstd{scores_json} \hlkwb{<-} \hlkwd{toJSON}\hlstd{(scores_df)}

\hlcom{## Convert the scores dataframe to JSON using the `toJSON()` function with the `pretty=TRUE` option}
\hlstd{scores_json2} \hlkwb{<-} \hlkwd{toJSON}\hlstd{(scores_df,}\hlkwc{pretty}\hlstd{=}\hlnum{TRUE}\hlstd{)}
\end{alltt}
\end{kframe}
\end{knitrout}

The R session information (including the OS info, R version and all
packages used):

\begin{knitrout}
\definecolor{shadecolor}{rgb}{0.969, 0.969, 0.969}\color{fgcolor}\begin{kframe}
\begin{alltt}
\hlkwd{sessionInfo}\hlstd{()}
\end{alltt}
\begin{verbatim}
## R version 4.3.1 (2023-06-16 ucrt)
## Platform: x86_64-w64-mingw32/x64 (64-bit)
## Running under: Windows 10 x64 (build 19045)
## 
## Matrix products: default
## 
## 
## locale:
## [1] LC_COLLATE=English_United States.utf8  LC_CTYPE=English_United States.utf8   
## [3] LC_MONETARY=English_United States.utf8 LC_NUMERIC=C                          
## [5] LC_TIME=English_United States.utf8    
## 
## time zone: America/New_York
## tzcode source: internal
## 
## attached base packages:
## [1] stats     graphics  grDevices utils     datasets  methods   base     
## 
## other attached packages:
## [1] DBI_1.1.3      readxl_1.4.2   jsonlite_1.8.5
## 
## loaded via a namespace (and not attached):
##  [1] utf8_1.2.3       fastmap_1.1.1    bit_4.0.5        xfun_0.39        cellranger_1.1.0
##  [6] magrittr_2.0.3   cachem_1.0.8     glue_1.6.2       blob_1.2.4       tibble_3.2.1    
## [11] knitr_1.43       memoise_2.0.1    pkgconfig_2.0.3  bit64_4.0.5      lifecycle_1.0.3 
## [16] tinytex_0.45     cli_3.6.1        RSQLite_2.3.1    fansi_1.0.4      vctrs_0.6.3     
## [21] compiler_4.3.1   highr_0.10       rstudioapi_0.14  tools_4.3.1      evaluate_0.21   
## [26] pillar_1.9.0     rlang_1.1.1
\end{verbatim}
\begin{alltt}
\hlkwd{Sys.time}\hlstd{()}
\end{alltt}
\begin{verbatim}
## [1] "2023-06-19 09:36:26 EDT"
\end{verbatim}
\end{kframe}
\end{knitrout}


\end{document}
